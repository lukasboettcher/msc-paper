
%-------------------------------------Packages-------------------------------------
% Language setting
\usepackage[english]{babel}

% Set page size and margins
\usepackage[a4paper,
            top = 2.5cm, 
            bottom = 2cm, 
            left = 3.5cm, 
            includehead, 
            includefoot, 
            right = 2.5cm]{geometry}

%Set headers and footers
\usepackage{fancyhdr}
\pagestyle{fancy}
\fancyhf{}                                  %Clear all footers and headers
%\fancyhead[L]{\leftmark}                    %Gives the chapter name

\renewcommand{\headrulewidth}{0.4pt}
\renewcommand{\footrulewidth}{0.4pt}
\renewcommand{\chaptermark}[1]{%
\markboth{\ \thechapter.\ #1}{}}
\renewcommand{\sectionmark}[1]{\markright{\thesection.\ #1}}

\setlength{\headheight}{15pt}

% % Useful packages
\usepackage{lipsum}                         %Create dummy text
\usepackage{setspace}                       %Give bigger spacing in between the lines of the title
\usepackage[nottoc]{tocbibind}                      %Add list of figures and list of tables to the toc
\usepackage[toc, acronym, nonumberlist, style = super]{glossaries}       %Use acronyms and create a section that shows the acronym and its meaning
\usepackage{chngcntr}                       %Change how figures are numbered
\usepackage{caption}                        
\usepackage{subcaption}                     %Allows subfigures
\usepackage{xcolor}
\usepackage{cmap} % allow copy paste from pdf
\usepackage[english]{babel}
\usepackage[utf8]{inputenc}
\usepackage[T1]{fontenc}
\usepackage{graphicx}
\usepackage{minted}
\usepackage[babel]{microtype}
\usepackage{mathtools}
\usepackage{xurl}
\usepackage[unicode]{hyperref}
\usepackage[math]{blindtext}
% \usepackage[mode=text,group-minimum-digits=4,round-mode=places,round-precision=2]{siunitx}
\usepackage{csquotes}
\usepackage{tikz}
\usepackage{pgfplots}
\pgfplotsset{compat=1.5}
\usepackage{algorithm,algpseudocode,algorithmicx}
\usepackage{multirow, makecell, bigstrut, booktabs, float}
\usepackage{tabularx}
\usepackage{amsmath}
\usepackage{amssymb}
\usepackage{bytefield}
\usepackage{setspace} 
\usepackage{listings}
\usepackage{emptypage}

\usepackage[
  backend       = biber,
  sortcites     = true,
  bibstyle      = alphabetic,
  citestyle     = alphabetic,
  giveninits    = true,
  useprefix     = false, %"von, van, etc." will be printed, too. See below.
  minnames      = 1,
  minalphanames = 3,
  maxalphanames = 4,
  maxbibnames   = 99,
  maxcitenames  = 2,
  natbib        = true,
  eprint        = true,
  url           = true,
  doi           = true,
  isbn          = true,
  backref       = true]{biblatex}
\bibliography{literatur}
\hypersetup{
    linktoc=all,
    bookmarksnumbered=true,
    bookmarksopen=true,
    bookmarksopenlevel=1,
    breaklinks=true,
    pdftitle={Masterarbeit},
    pdfstartview=Fit,
    pdfpagelayout=SinglePage,    
}
\usetikzlibrary{arrows.meta,fit,calc,decorations.pathreplacing,calligraphy}
\emergencystretch=1em
\newcommand{\listingautorefname}{Listing}
\newcommand{\algorithmautorefname}{Algorithm}
% \memsection{end address}{start address}{height in lines}{text in box}
\newcommand{\memsection}[4]{%
    % define the height of the memsection
    \bytefieldsetup{bitheight=#3\baselineskip}%
    \bitbox[]{10}{%
        \texttt{#1}% print end address
        \\
        % do some spacing
        \vspace{#3\baselineskip}
        \vspace{-2\baselineskip}
        \vspace{-#3pt}
        \texttt{#2}% print start address
    }%
    \bitbox{16}{#4}% print box with caption
}

\fancyhead[L]{\textit{\rightmark}}                   %Gives the section name
\fancyhead[R]{\myName{}}
\fancyfoot[C]{\thepage}


% \input{auxfiles/code_listings}             %Defining SCL as a programming language for use in listing


\setlength{\parindent}{0pt}                 %No indentation in new paragraphs
\setlength{\parskip}{6pt}                   %Distance between paragraphs = 6pt
\renewcommand{\baselinestretch}{1.3}        %Line spacing = 1.3

%\counterwithout{figure}{chapter}            %Change figure numbering to sequential order (1,2,...,n)
%\counterwithout{table}{chapter}

% \makeglossaries

% \graphicspath{{figures/}}

\def\code#1{\textit{\texttt{#1}}}                    %Define a command to write monospace text in form of a code

%Define Pagestyles
\fancypagestyle{plain}{
    \fancyhf{}
    \fancyhead[R]{\myName{}}
    \fancyfoot[C]{\thepage}
    \renewcommand{\headrulewidth}{0.4pt}
    \renewcommand{\footrulewidth}{0.4pt}
}